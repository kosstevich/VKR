% вторая часть

\section{Проектирование приложения с учётом внесённых предложений}

\subsection{Требования к проектируемому приложению}\label{trebovaniya}

С целью обеспечения наибольшей точности и минимизации рисков,
связанных с возможностью некорректного принятия решений относительно
герметичности ТВС, предлагаю разработать приложение для обработки и анализа данных, полученных при КГО стендовым водным методом на реакторе типа ВВЭР, которое призвано помочь лицу, принимающему решения относительно
герметичности ТВС. Основной задачей данного приложения является автоматизированная(под контролем ЛПР) обработка и анализ данных. Исходные данные хранятся в табличном виде в формате ODS. Под обработкой и анализом данных понимается разделение исходных данных на выборки, принадлежащие к
одному статистическому распределению, произведение всех необходимых
расчётов, а также принятие решения относительно герметичности ТВС. После предварительного анализа методики, описанной в главе 1, а также учитывая изложенное в главе 2, могу выдвинуть следующие требования к приложению:

1. Приложение должно наглядно демонстрировать ЛПР основания принятия
решения относительно герметичности ТВС, т.е. иметь обширный функционал
для визуализации и экспорта графиков распределений, негерметичных ТВС и т.д. Исходя из этого был выбран формат настольного приложения с графическим интерфейсом.

2. Приложение должно иметь функционал для проведения статистических тестов с целью проверки выборок, заданных пользователем, на принадлежность одному статистическому распределению, а также на характер распределения.

В качестве статистических тестов предлагаю использовать:
\begin{itemize}
	\item U-критерий Манна-Уитни --- c целью установления принадлежности выборок к одному статистическому распределению
	\item Критерий Шапиро-Уилка~\cite{RANstat} --- с целью проверки на отклонение распределения выборки от нормального закона.
	\item Критерий Стьюдента для независимых выборок~--- c целью установления статистически значимых различий между независимыми выборками, независимые выборки должны быть распределены нормально (Проверяется критерием Шапиро-Уилка).
\end{itemize}

3. Приложение должно поддерживать поиск выбросов методами "IQR"\ и "3 cигма". По усмотрению ЛПР может применяться любой метод в зависимости от ситуации.

4. Приложение должно иметь возможности для экспорта всех преобразованных
данных, на основании которых принимались решения о герметичности, в
исходный ODS формат.

\subsection{Архитектура приложения}

На данном этапе проектирования выделяются независимые модули, из которых будет состоять проектируемое приложение:

1. Модуль пользовательского интерфейса, который отвечает за создание
графического интерфейса, позволяющего пользователям взаимодействовать с другими модулями системы. Данный модуль поддерживает функции работы с графиками, в том числе редактирование, создание легенды, а также сохранение в формате картинки для последующего использования в отчётах о проведении КГО.

2. Модуль предварительной обработки(подготовки), который
отвечает за импорт и очистку исходных данных.

3. Модуль обработки и анализа данных, который отвечает за проведение
расчётов и статистического анализа, полученных после предварительной
обработки. Включает в себя выполнение статистических тестов, поиск выбросов, а также экспорт полученных данных в .ods формат.

\subsection{Пользовательский сценарий использования} \label{Scenarii}

Ниже приведены шаги, которые будет выполнять пользователь при работе с проектируемым приложением:

1. Импорт данных: пользователь выбирает файл в формате .ods с входными данными для анализа.

2. Анализ входных данных: перед пользователем открывается окно, в котором имеется возможность построения и гибкого редактирования графиков по входным данным. Согласно, 1.3.3 на основании визуального анализа графиков, пользователь разделяет входные данные на выборки.

3. Анализ выборок: После разбиения входных данных возникает возможность выполнения статистических тестов для каждой выборки. Пользователь выбирает выборку и анализируемую величину, после чего запускает статистические тесты. Результаты тестов выводятся в отдельном окне для каждой выборки и анализируемой величины. На основании результатов тестов, пользователь делает заключение о корректности разбиения входных данных и возможности дальнейшего анализа.

В случае, если результаты статистических тестов не позволяют проводить дальнейший поиск выбросов в выборках, окно с выборками закрывается и шаги 2-3 повторяются до тех пор, пока все выборки не будут удовлетворять условиям, позволяющим проводить поиск выбросов.

4. После проведения статистических тестов для выборок пользователь запускает поиск выбросов в выборках согласно выбранному методу. Результатом поиска является 3 таблицы:
\begin{itemize}
	\item Таблица негерметичных ТВС
	\item Таблица ТВС, для которых необходимо провести повторную проверку
	\item Таблица параметров, на основании которых принималось заключение о герметичности
\end{itemize}

5. После проведения анализа пользователь экспортирует полученные таблицы в .ods для составления отчёта о проведении КГО.