% вторая часть

\section{Проектирование приложения с учётом внесённых предложений}

\subsection{Требования к проектируемому приложению}\label{trebovaniya}

С целью обеспечения наибольшей точности и минимизации рисков,
связанных с возможностью некорректного принятия решений относительно
герметичности ТВС, предлагаю разработать приложение для обработки и анализа данных, полученных при КГО стендовым водным методом на реакторе типа ВВЭР, которое призвано помочь лицу, принимающему решения относительно
герметичности ТВС. Основной задачей данного приложения является автоматизированная(под контролем ЛПР) обработка и анализ данных. Исходные данные хранятся в табличном виде в формате ODS. Под обработкой и анализом данных понимается разделение исходных данных на выборки, принадлежащие к
одному статистическому распределению, произведение всех необходимых
расчётов, а также принятие решения относительно герметичности ТВС. После предварительного анализа методики, описанной в главе 1, а также учитывая изложенное в главе 2, могу выдвинуть следующие требования к приложению:

1. Приложение должно наглядно демонстрировать ЛПР основания принятия
решения относительно герметичности ТВС, т.е. иметь обширный функционал
для визуализации и экспорта графиков распределений, негерметичных ТВС и т.д. Исходя из этого был выбран формат настольного приложения с графическим интерфейсом.

2. Приложение должно иметь функционал для проведения статистических тестов с целью проверки выборок, заданных пользователем, на принадлежность одному статистическому распределению, а также на характер распределения.

В качестве статистических тестов предлагаю использовать:
\begin{itemize}
	\item U-критерий Манна-Уитни --- c целью установления принадлежности выборок к одному статистическому распределению
	\item Критерий Шапиро-Уилка~\cite{RANstat} --- с целью проверки на отклонение распределения выборки от нормального закона.
	\item Критерий Стьюдента для независимых выборок~--- c целью установления статистически значимых различий между независимыми выборками, независимые выборки должны быть распределены нормально (Проверяется критерием Шапиро-Уилка).
\end{itemize}

3. Приложение должно поддерживать поиск выбросов методами "IQR"\ и "3 cигма". По усмотрению ЛПР может применяться любой метод в зависимости от ситуации.

4. Приложение должно иметь возможности для экспорта всех преобразованных
данных, на основании которых принимались решения о герметичности, в
исходный ODS формат.

\subsection{Обзор инструментов разработки}

Для разработки приложения на основании требований, описанных в \ref{trebovaniya}, были выбраны следующие инструменты:

1. Язык программирования Python - высокоуровневый, интерпретируемый, объектно-ориентированный язык программирования, который широко используется для разработки веб-приложений, научных вычислений, анализа данных, искусственного интеллекта, автоматизации задач и многих других областей. В совокупности с большим набором пользовательских библиотек, python предоставляет мощные инструменты для обработки, анализа и визуализации данных. 

В качестве альтернативы Python рассматривался язык R. R предоставляет более широкий спектр функций по обработке и анализу данных, но набор инструментов для визуализации, а также создания интерфейса приложений ограничен. Именно этот фактор стал решающим в пользу Python.

2. Библиотека Pandas -  программная библиотека на языке Python для обработки и анализа данных. Работа Pandas с данными строится поверх библиотеки NumPy, являющейся инструментом более низкого уровня. Предоставляет специальные структуры данных и операции для манипулирования табличными данными. Преимущество Pandas заключается в дешевизне операций и скорости работы, что делает ее неотъемлемым инструментом для анализа данных, машинного обучения, статистики и других областей, где требуется работа с табличными данными.

3. Библиотека NumPy предоставляет эффективные контейнеры для
работы с массивами и матрицами данных. В совокупности с Pandas она широко используется для выполнения математических операций и вычислений в Python.

4. SciPy --- библиотека, основанная на расширении NumPy, которая применяется для более сложных научных и инженерных вычислений. SciPy в основном написана на Python и частично на языках C, C++ и Fortran, в связи с чем отличается высокой производительностью и скоростью работы. В рамках разработки приложения использовался модуль scipy.stats, который предоставляет обширный функционал для проведения статистических вычислений.

5. Библиотеки Matplotlib и Seaborn. Эти библиотеки предоставляют
возможности для визуализации данных в Python. Matplotlib является основной
библиотекой для создания различных типов графиков, в то время как Seaborn
предоставляет более высокоуровневый интерфейс для создания
статистических графиков.

6. PyQt --- набор расширений кроссплатформенного графического фреймворка Qt, выполненный в виде библиотеки Python. Qt --- фреймворк для разработки кроссплатформенного программного обеспечения c графическим интерфейсом, написанный на языке программирования C++.

\subsection{Архитектура приложения}

\subsection{Пользовательский сценарий использования}

Ниже приведены шаги, которые будет выполнять пользователь при работе с проектируемым приложением:

1. Импорт данных: пользователь выбирает файл в формате .ods с входными данными для анализа.

2. Анализ входных данных: перед пользователем открывается окно, в котором имеется возможность построения и гибкого редактирования графиков по входным данным. Согласно, 1.3.3 на основании визуального анализа графиков, пользователь разделяет входные данные на выборки.

3. Анализ выборок: После разбиения входных данных возникает возможность выполнения статистических тестов для каждой выборки. Пользователь выбирает выборку и анализируемую величину, после чего запускает статистические тесты. Результаты тестов выводятся в отдельном окне для каждой выборки и анализируемой величины. На основании результатов тестов, пользователь делает заключение о корректности разбиения входных данных и возможности дальнейшего анализа.

В случае, если результаты статистических тестов не позволяют проводить дальнейший поиск выбросов в выборках, окно с выборками закрывается и шаги 2-3 повторяются до тех пор, пока все выборки не будут удовлетворять условиям, позволяющим проводить поиск выбросов.

4. 