% пример введения

Атомные электростанции играют ключевую роль в современной
энергетике. Однако сопутствующие ядерной энергетике риски требуют
непрерывного совершенствования методов контроля и обслуживания ядерных установок.

В частности, одним из значимых аспектов эксплуатации ядерных реакторов является контроль герметичности оболочек тепловыделяющих
элементов. В настоящее время анализ данных, полученных при проведении
КГО, частично осуществляется в ручном режиме, что требует значительных ресурсов времени и труда. Более того, этот подход подвержен человеческим ошибкам и может ограничивать возможности в проведении анализа данных с высокой точностью и скоростью.

Как известно, одним из недостатков реактора типа ВВЭР является
невозможность перегрузки топлива без остановки реактора и ошибка,
допущенная при принятии решения относительно герметичности ТВС, может
потребовать незапланированную остановку реактора, что влечёт за собой
существенные экономические издержки.

Цель настоящей работы заключается в разработке программного обеспечения, работа которого направлена на повышение эффективности
и достоверности результатов КГО, а также снижение трудовых затрат.

В данной работе будет проведен обзор существующего метода обработки результатов КГО, приведены предложения по его автоматизации, а также описан процесс создания прототипа программного обеспечения.
%\cite{SoetaertRJ2010}.
%\cite{kotelnikov} 