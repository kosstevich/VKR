% вторая часть

\section{Разработка приложения}

\subsection{Исходный код}

Исходный код программ можно добавить с помощью окружений, определенных сразу после преамбулы. Пример --- на листинге~\ref{lst:1}.


\begin{flushleft}
\needspace{3\baselineskip}
\captionof{Program}{Пример кода на языке R} \label{lst:1}
\begin{MyCodes}
# Проверка и тестирование пакета deSolve
require(deSolve)
require(rgl)

# система Хиндмарша - Розе с параметрами
# используются параметры в виде списка (parms$a etc)
hindrose <- function(t,y,parms)
{
 ydot <- vector(len=3)
 ydot[1] <- y[2] - parms$a * y[1]^3 + parms$b*y[1]^2 + parms$Iext - y[3]
 ydot[2] <- parms$c - parms$d*y[1]^2 - y[2]
 ydot[3] <- parms$r * (parms$s*(y[1]-parms$xs)-y[3])
 return(list(ydot))
}
\end{MyCodes}
\end{flushleft}