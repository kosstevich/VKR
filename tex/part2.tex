% вторая часть
\section{Предложения по улучшению методики обработки данных}

\subsection{Проблемы существующего подхода}
Рассмотрев методику проведения процедуры КГО согласно\cite{RD} можно выделить несколько замечаний, которые можно пересмотреть:

1. Метод анализа данных, приведённый в параграфе 1.3, основывается на поиске выбросов по правилу "3 сигм". Данное правило утверждает, что  абсолютная величина отклонения нормально распределённой случайной величины от её математического ожидания не превосходит трёх среднеквадратичных отклонений с вероятностью\cite{KremerMatstat}:
\begin{equation} \label{eq:3sigma_rule}
	P(|X - m| < 3\sigma)  = 0,9973
\end{equation}

Проблема этого метода заключается в том, что он применим для выборок, значения которых извлечены из нормально распределённых генеральных совокупностей. Но в случае проведения КГО(согласно разделу 1) нет достаточных оснований утверждать, что все значения активностей будут распределены по нормальному закону. Следовательно, требуется проверить характер распределения для каждой выборки с целью установления корректности применения метода "3 сигм".

Кроме того, среднее и среднеквадратическое отклонение, рассчитываемые в данном методе, также изменяются под воздействием аномальных значений, что приводит к маскировке выбросов\cite{emissions}. Следовательно, имеет смысл рассмотреть альтернативы, например метод межквартильного размаха(IQR).

2. Процедура КГО с учётом времени и объёма испытаний может проходить до нескольких недель. С течением времени в БВ, а также в воде, подаваемой на стенд КГО, может изменяться концентрация борной кислоты с целью борного регулирования, что негативно сказывается на однородности условий проведения испытаний. В связи с этим значения активностей ПД, полученные в разное время, могут принадлежать разным статистическим распределениям, следовательно, анализироваться должны отдельно. 

Согласно пункту 1.3.3, разделение на выборки происходит "На основании визуального анализа"\ графических данных. Хочу отметить, что в изучаемой методике существует способ анализа полученных выборок на принадлежность к одному статистическому распределению с целью объединения нескольких выборок, но он не учитывает анализ на корректность разбиения исходных данных. В связи с вышеперечисленным возникает необходимость проверки данных в выборках, полученных на основании визуального анализа.

3. Описанный процесс анализа данных производится в ручном режиме с использованием программного комплекса Excel. Автоматизация этого процесса позволит снизить вероятность ошибок, ведущих к преждевременной остановке реактора, а также снизить потребность во временных и трудовых затратах.

\subsection{Метод IQR}

\subsection{Анализ принадлежности данных одному статистическому распределению}


% 3. Исходя из изложенного в 1.3.1-1.3.2 требуется разделять исходные данные на выборки, принадлежащие как минимум одному статистическому распределению, в идеальном случае требуется, чтобы выборка подчинялась нормальному закону распределения.