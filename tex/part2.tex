% вторая часть
\section{Предложения по улучшению методики обработки данных}

\subsection{Проблемы существующего подхода}
Рассмотрев методику проведения процедуры КГО согласно\cite{RD} можно выделить несколько замечаний, которые можно пересмотреть:

1. Метод анализа данных, приведённый в параграфе 1.3, основывается на поиске выбросов по правилу "3 сигм". Данное правило утверждает 

2. Процедура КГО с учётом времени и объёма испытаний может проходить до нескольких недель. Кроме того, с течением времени может изменяться концентрация борной кислоты в БВ и в воде, подаваемой на стенд КГО с целью борного регулирования. Как известно 
В связи с этим значения активностей ПД, полученные в разное время, могут принадлежать разным статистическим распределениям.

% Этот подход применим для выборок, значения которых извлечены из нормально распределённых генеральных совокупностей. Исходя из опыта эксплуатации НВ АЭС, значения активностей крайне редко бывают распределены по нормальному закону в силу влияния множества факторов. Кроме того, среднее и среднеквадратическое отклонение, рассчитываемые в данном методе, также изменяются под воздействием аномальных значений, что приводит к маскировке выбросов\cite{emissions}. Следовательно, имеет смысл рассмотреть альтернативы методу "трёх сигм". Предложения на замену данному приведены в параграфе [].

% 3. Исходя из изложенного в 1.3.1-1.3.2 требуется разделять исходные данные на выборки, принадлежащие как минимум одному статистическому распределению, в идеальном случае требуется, чтобы выборка подчинялась нормальному закону распределения.