% первая часть

\section{Глава 1 Обзор существующей методики проведения процедуры КГО стендовым методом}

\subsection{Основные положения}

В данной работе рассматривается метод КГО в пеналах СОДС, который является одним из наиболее надёжных
способов определения негерметичных ТВС. СОДС входит в состав обязательного оборудования всех действующих и проектируемых АЭС с реактором ВВЭР.

Метод основан на измерении утечки ПД из-под оболочек твэлов путем гамма-спектрометрического анализа изотопного состава проб воды, отбираемых из контура циркуляции СОДС, по активности реперных радионуклидов $^{131}$I, $^{134}$Cs, $^{136}$Cs, $^{137}$Cs и $^{133}$Xe. Инициирование выхода радионуклидов в воду стенда КГО осуществляется посредством изменения давления циркулирующей по контуру стенда воды в процессе выдержки ТВС в этой воде -- настаивании.


\subsection{Процедура проведения КГО стендовым методом}
1. Процедура проведения КГО начинается проведения испытаний для каждой ТВС в пеналах СОДС с последующим отбором проб воды. 

Проверка ТВС проводится при циркуляции воды по контуру стенда КГО без ее замены и состоит из двух циклов:

- Настаивание ТВС при изыбыточном (верхнем) давлении в контуре от 4,5 * 10$^{5}$ Па до 6,0 * 10$^{5}$ Па продолжительностью 5 минут.

- Настаивание ТВС при избыточном (нижнем) давлении в контуре от 1,0 * 10$^{5}$ Па до 1,5 * 10$^{5}$ Па до полного перемешивания (не менее 15 минут).

С целью соблюдения одинаковых условий испытаний требуется, чтобы значения верхнего и нижнего избыточного давления были одинаковыми при проверке всех ТВС. 

2. После завершения настаивания ТВС производится отбор пробы воды из контура стенда КГО.

3. В каждой $j$-ой пробе воды, взятой из стенда КГО при испытании $j$-ой ТВС, на спектрометрической установке измеряются значения удельной активности и приводятся на момент останова реактора:

- $A_{j,кго}^{i}$ --- реперных $i$-х радионуклидов продуктов деления ($^{131}$I, $^{134}$Cs, $^{136}$Cs, $^{137}$Cs и $^{133}$Xe)

- $A_{j,кго}^{i'}$ --- радионуклида продукта коррозии ($^{54}$Mn или $^{58}$Co, $^{60}$Co, $^{51}$Cr, $^{59}$Fe).

4. Для учета фоновой активности радионуклидов йода, цезия и
продуктов коррозии периодически производится измерение их активности
в воде, подаваемой в стенд КГО (с каждой вновь приготовленной порцией
раствора борной кислоты на СВО), и в бассейне выдержки (один раз в
сутки).

5. Проверка фоновой составляющей за счет загрязнения стенда
радиоактивными продуктами (холостая проба) производится перед началом
работ по КГО, а также периодически (не реже одного раза в сутки). Для этого
без загрузки ТВС в пенал проводятся все операции по промывке контура и
настаиванию с отбором и анализом пробы.

6. Итогом проведения спектрометрического анализа проб воды является
таблица значений, в которых для каждой $j$-ой ТВС приводятся в соответствие
значения активности $A_{j,кго}^{i}$ каждого из регистрируемых реперных радионуклидов
и $A_{j,кго}^{i'}$ продуктов коррозии. Статистический анализ результатов
измерения проводится для ТВС, в пробах которых значимо регистрировались
ПД. Результаты измерений ТВС, при проверке которых реперные ПД не
регистрировались, из статистического расчета исключаются.

\subsection{Обработка результатов}
1. Анализ герметичности ТВС основан на выборочном поиске выбросов методом "3 сигм". Зачастую данный метод наиболее показателен в выборках, извлечённых из нормально распределённой генеральной совокупности. Однако этот метод(так же как и обратный ему "Z оценка") устойчив и для других видов распределения.

2. Процедура КГО, описанная в параграфе 1.2, с учётом времени и объёма испытаний может проходить до нескольких недель. Кроме того, с течением времени может изменяться концентрация борной кислоты в БВ и в воде, подаваемой на стенд КГО. В связи с этим значения активностей ПД, полученные в разное время, могут принадлежать разным статистическим распределениям.

3. Исходя из изложенного в 1.3.1-1.3.2 требуется разделять исходные данные на выборки, принадлежащие одному статистическому распределению. Поиск негерметичных ТВС происходит в каждой выборке раздельно. 

\subsection{Формулы}

Формулы в \LaTeXe\ выглядят достаточно красиво, как строчные --- $E=mc^2$, так и выключные (см. \ref{eq:pi}):

\begin{equation} \label{eq:pi}
\pi = \int\limits_0^1 \frac{4}{1+x^2} dx
\end{equation}