% первая часть

\section{Обзор существующей методики проведения процедуры КГО стендовым методом}

\subsection{Основные положения}

В данной работе рассматривается метод КГО в пеналах СОДС\cite{RD}, который является одним из наиболее надёжных способов определения негерметичных ТВС. СОДС входит в состав обязательного оборудования всех действующих и проектируемых АЭС с реактором ВВЭР.

Метод основан на измерении утечки ПД из-под оболочек твэлов путем гамма-спектрометрического анализа изотопного состава проб воды, отбираемых из контура циркуляции СОДС, по активности реперных радионуклидов $^{131}$I, $^{134}$Cs, $^{136}$Cs, $^{137}$Cs и $^{133}$Xe. Инициирование выхода радионуклидов в воду стенда КГО осуществляется посредством изменения давления циркулирующей по контуру стенда воды в процессе выдержки ТВС в этой воде -- настаивании.


\subsection{Процедура проведения КГО стендовым методом}
1. Процедура проведения КГО начинается проведения испытаний для каждой ТВС в пеналах СОДС с последующим отбором проб воды. 

Проверка ТВС проводится при циркуляции воды по контуру стенда КГО без ее замены и состоит из двух циклов:
\begin{itemize}
\item Настаивание ТВС при изыбыточном (верхнем) давлении в контуре от 4,5 * 10$^{5}$ Па до 6,0 * 10$^{5}$ Па продолжительностью 5 минут.

\item Настаивание ТВС при избыточном (нижнем) давлении в контуре от 1,0 * 10$^{5}$ Па до 1,5 * 10$^{5}$ Па до полного перемешивания (не менее 15 минут).
\end{itemize}

С целью соблюдения одинаковых условий испытаний требуется, чтобы значения верхнего и нижнего избыточного давления были одинаковыми при проверке всех ТВС. 

2. После завершения настаивания ТВС производится отбор пробы воды из контура стенда КГО.

3. В каждой $j$-ой пробе воды, взятой из стенда КГО при испытании $j$-ой ТВС, на спектрометрической установке измеряются значения удельной активности и приводятся на момент останова реактора:
\begin{itemize}
\item $A_{j,кго}^{i}$ --- реперных $i$-х радионуклидов продуктов деления ($^{131}$I, $^{134}$Cs, $^{136}$Cs, $^{137}$Cs и $^{133}$Xe)

\item $A_{j,кго}^{i'}$ --- радионуклида продуктов коррозии(ПК) ($^{54}$Mn или $^{58}$Co, $^{60}$Co, $^{51}$Cr, $^{59}$Fe).
\end{itemize}

4. Для учета фоновой активности радионуклидов йода, цезия и
продуктов коррозии периодически производится измерение их активности
в воде, подаваемой в стенд КГО (с каждой вновь приготовленной порцией
раствора борной кислоты на СВО), и в бассейне выдержки (один раз в
сутки).

5. Проверка фоновой составляющей за счет загрязнения стенда
радиоактивными продуктами (холостая проба) производится перед началом
работ по КГО, а также периодически (не реже одного раза в сутки). Для этого
без загрузки ТВС в пенал проводятся все операции по промывке контура и
настаиванию с отбором и анализом пробы.

6. Итогом проведения спектрометрического анализа проб воды является
таблица значений, в которых для каждой $j$-ой ТВС приводятся в соответствие
значения активности $A_{j,кго}^{i}$ каждого из регистрируемых реперных радионуклидов
и $A_{j,кго}^{i'}$ продуктов коррозии. Статистический анализ результатов
измерения проводится для ТВС, в пробах которых значимо регистрировались
ПД. Результаты измерений ТВС, при проверке которых реперные ПД не
регистрировались, из статистического расчета исключаются.

\subsection{Обработка результатов}
1. Анализ герметичности ТВС, согласно \cite{RD}, основан на выборочном поиске выбросов методом "трёх сигм". Этот подход применим для выборок, значения которых извлечены из нормально распределённых генеральных совокупностей. Исходя из опыта эксплуатации НВ АЭС, значения активностей крайне редко бывают распределены по нормальному закону в силу влияния множества факторов. Кроме того, среднее и среднеквадратическое
отклонение, рассчитываемые в данном методе, также изменяются под воздействием аномальных значений, что приводит к маскировке выбросов\cite{emissions}. Следовательно, имеет смысл рассмотреть альтернативы методу "трёх сигм". Предложения на замену данному приведены в параграфе [].

% 2. Процедура КГО, описанная в параграфе 1.2, с учётом времени и объёма испытаний может проходить до нескольких недель. Кроме того, с течением времени может изменяться концентрация борной кислоты в БВ и в воде, подаваемой на стенд КГО. В связи с этим значения активностей ПД, полученные в разное время, могут принадлежать разным статистическим распределениям.

% 3. Исходя из изложенного в 1.3.1-1.3.2 требуется разделять исходные данные на выборки, принадлежащие как минимум одному статистическому распределению, в идеальном случае требуется, чтобы выборка подчинялась нормальному закону распределения.

2. Основными реперными радионуклидами, по которым устанавливается наличие(отсутствие) негерметичных твэлов в ТВС являются $^{131}$I, $^{134}$Cs, $^{137}$Cs. Наличие в контролируемой пробе $^{136}$Cs и(или) $^{133}$Xe, значимо превышающих их содержание в холостых пробах, является однозначным основанием для включение ТВС в список подозрительных, требующих как минимум дополнительной проверки.

3. Полученные значения представляются в графическом виде в такой
хронологической последовательности, в какой ТВС проверялись в стенде
КГО. Примеры графического представления результатов КГО приведены на
рисунках~\ref{fig:ris1} и~\ref{fig:ris2}. На основании визуального анализа этих данных на графике
может быть сделано заключение, относятся ли они к одному статистическому
распределению. Таким способом проводится оценка соблюдения одинаковых
условий проверки всех ТВС. Если условия менялись с течением времени (на
практике так происходит почти всегда), то производится разделение исходных данных на выборки, которые относятся к одному статистическому распределению.

\begin{figure}[H]
	\centering
	\includegraphics[width=0.5\linewidth]{pics/ris1} % изображения хранятся в подкаталоге pics
	\caption{Графическое хронологическое представление данных, принадлежащих к одному распределению.\cite{RD}}
	\label{fig:ris1} % эта метка позволяет ссылаться на рисунок в тексте
\end{figure}

\begin{figure}[H]
	\centering
	\includegraphics[width=0.5\linewidth]{pics/ris1} % изображения хранятся в подкаталоге pics
	\caption{Графическое хронологическое представление данных, принадлежащих к различным распределениям.\cite{RD}}
	\label{fig:ris2} % эта метка позволяет ссылаться на рисунок в тексте
\end{figure}

4. Для каждой полученной совокупности данных, относящихся к одному
и тому же статистическому распределению, вычисляются $\overline{A}_{кго}^{i}$ --- среднеарифметические значения удельной активности радионуклидов $^{131}$I (и $^{134}$Cs, $^{136}$Cs, $^{137}$Cs, $^{133}$Xe) и $\overline{A}_{кго}^{i'}$ --- среднеарифметическое значение удельной активности $^{54}$Mn (или $^{58}$Co, $^{60}$Co, $^{51}$Cr, $^{59}$Fe), по формулам~\ref{eq:Mean} и~\ref{eq:corosionMean}:

\begin{equation} \label{eq:Mean}
	\overline{A}_{кго}^{i} = \frac{1}{N}\sum_{j=1}^{N}A_{j,кго}^{i}
\end{equation}

\begin{equation} \label{eq:corosionMean}
	\overline{A}_{кго}^{i'} = \frac{1}{N}\sum_{j=1}^{N}A_{j,кго}^{i'}
\end{equation}

Кроме того, рассчитывают соответствующие им среднеквадратичные
отклонения (стандартные статистические неопределенности) по формулам~\ref{eq:Std} и~\ref{eq:corosionStd}:

\begin{equation} \label{eq:Std}
	\sigma_{{A}_{кго}^{i}} = \sqrt{\frac{1}{N-1}\sum_{j=1}^{N}(A_{j,кго}^{i} - \overline{A}_{кго}^{i})^2}
\end{equation}
	
\begin{equation} \label{eq:corosionStd}
	\sigma_{{A}_{кго}^{i'}} = \sqrt{\frac{1}{N-1}\sum_{j=1}^{N}(A_{j,кго}^{i'} - \overline{A}_{кго}^{i'})^2}
\end{equation}

где N --- количество проверенных ТВС.

5. Если N > 10, то ТВС, для которых выполняется условие~\ref{eq:3sigma} являются герметичными.

\begin{equation} \label{eq:3sigma}
	A_{j,кго}^{i} \leq {A}_{кго}^{i} + 3*\sigma_{{A}_{кго}^{i}}
\end{equation}

ТВС, для которых одновременно выполняются условия~\ref{eq:3sigma_non_hermetic} и~\ref{eq:3sigma_corosion} являются негерметичными.

\begin{equation} \label{eq:3sigma_non_hermetic}
	A_{j,кго}^{i} > {A}_{кго}^{i} + 3*\sigma_{{A}_{кго}^{i}}
\end{equation}

\begin{equation} \label{eq:3sigma_corosion}
	A_{j,кго}^{i'} \leq {A}_{кго}^{i'} + 3*\sigma_{{A}_{кго}^{i'}}
\end{equation}

Важно отметить, что активности радионуклидов ПК измеряются с целью учёта при анализе данных. ПК, образующиеся в конструкционных материалах реактора по мере эксплуатации, переносятся по теплоносителю и могут откладываться на ТВС, что влечёт за собой повышение активности в том числе и реперных ПД\cite{corosion}. Именно поэтому повышение активности реперных ПД совместно с активностями ПК может являться признаком некачественной отмывки ТВС при подготовке к проведению испытаний.

6. Если количество ТВС в выборке N < 10, то в формулах~\ref{eq:3sigma}-\ref{eq:3sigma_corosion}  в качестве коэффициента при  и  вместо коэффициента 3 используются коэффициенты Стьюдента, приведенные в таблице~\ref{tab:Student}, для доверительной вероятности 0,95.
\begin{table}[H]
	\caption{Значения коэффициента Стьюдента в зависимости от количества проверенных ТВС и вероятности, с которой ТВС могут быть отнесены к разряду имеющих негерметичные твэлы} \label{tab:Student}
	\centering
	\begin{tabular}{|c|c|c|c|}
		\hline Кол-во ТВС & 0,95 & 0,99 & 0,999 \\ 
		\hline  2 & 12,7 & 66,7  & 637 \\ 
		\hline 3 &  4,30 &  9,93 & 31,6 \\ 
		\hline 4 &  3,18 &  5,84 & 12,9 \\ 
		\hline 5 &  2,78 &  4,60 & 8,61 \\ 
		\hline 6 &  2,57 &  4,03 & 6,86 \\ 
		\hline 7 &  2,45 &  3,71 & 5,96 \\ 
		\hline 8 &  2,36 &  3,50 & 5,41 \\ 
		\hline 9 &  2,31 &  3,36 & 5,04 \\ 
		\hline 10 & 2,26 &  3,25 & 4,78 \\ 
		\hline 
	\end{tabular} 
\end{table} 

